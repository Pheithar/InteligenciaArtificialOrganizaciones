\documentclass[12pt,a4paper, xcolor=table]{article}
\usepackage{graphicx}
\usepackage[utf8]{inputenc}
\usepackage{eurosym}
\usepackage[spanish,es-tabla]{babel}
\usepackage[left=2cm, right=2cm, top=2cm, bottom=2cm]{geometry}
\usepackage{afterpage}
\PassOptionsToPackage{hyphens}{url}\usepackage{hyperref}
\usepackage{subfig}
\usepackage[table,xcdraw]{xcolor}


\usepackage{imakeidx}
\newcommand\blankpage{%
    \null
    \thispagestyle{empty}%
    \addtocounter{page}{-1}%
    \newpage}
\renewcommand*\contentsname{Índice: }

\makeindex
\let\olditemize\itemize
\def\itemize{\olditemize\itemsep=0pt}

\begin{document}
\setlength{\parindent}{0pt}
\begin{titlepage}
        \centering
        \includegraphics[width=0.75\textwidth]{img/logo_uc3m.jpg}\par\vspace{2cm}
        {\huge\bfseries Práctica 2 \\ Análisis de las reseñas de Tripadvisor\par}
        \vspace{0.5cm}
        {\scshape\Large Inteligencia Artificial en las Organizaciones\par}
        \vspace{1.5cm}
        {\scshape\Large Grupo 83-1\par}
        \vspace{1.5cm}
        {\Large\itshape Miguel Gutiérrez Pérez\par}
        {\Large 100383537@alumnos.uc3m.es \par}
        \vspace{1cm}
        {\Large\itshape Mario Lozano Cortés\par}
        {\Large 100383511@alumnos.uc3m.es\par}
        \vspace{1cm}
        {\Large\itshape Alba Reinders Sánchez\par}
        {\Large 100383444@alumnos.uc3m.es\par}
        \vspace{1cm}
        {\Large\itshape Alejandro Valverde Mahou\par}
        {\Large 100383383@alumnos.uc3m.es\par}
        \vspace{5mm}
        {\large GitHub: \textbf{\textit{\href{https://github.com/Pheithar/InteligenciaArtificialOrganizaciones}{InteligenciaArtificialOrganizaciones}}}}
        \vfill

% Bottom of the page
        {\large \today\par}
\end{titlepage}

\tableofcontents

\newpage

\section{Introducción}
    La siguiente sección incluye 
    
\section{Parte 1: Clasificación}
    \subsection{Análisis y preprocesado de datos}
        Explicación de los pasos previos de lo que vimos en clase y división en cat
    \subsection{Experimentación}
        Generalidades de todos los experimentos. hacer hincapié en lo de las stopwords. Caso base
        \subsubsection{Experimentación básica}
        Tablas de cada uno de lo que hicimos esa tarde
        
        \subsubsection{Experimentación avanzada}
        Combinación de los mejores resultados básicos
        
    \subsection{Comentario de los resultados obtenidos}
    Mucho text

\section{Parte 2: Clustering}
 10 modelos de K Medias variando n Clústeres, semillas y funciones de distancia

Analizar al menos 1 (en detalle)

clasificación sobre el modelo y sacar el árbol

\section{Conclusiones}


\clearpage 

\section{Referencias}
    \begin{itemize}
        \item [1.] Introduction to Neurons in Neural Networks. Medium. Consultado en Octubre 2020. Url: \\
        \href{https://medium.com/artificial-neural-networks/introduction-to-neurons-in-neural-networks-71828d040a65}{https://medium.com/artificial-neural-networks}
    \end{itemize}
\printindex



  \section{Anexos}
  \begin{itemize}
    \item [1.] Perceptron Multicapa usando '\textit{K Fold}'\\
    \textbf{\textit{perceptron\_kfold.py}}
    \item [2.] Perceptron Multicapa usando '\textit{split percentage}'\\
    \textbf{\textit{perceptron\_split.py}}
    \item [3.] Programa para realizar la predicción de los modelos\\
    \textbf{\textit{predict.py}}
    \item [4.] Tabla de resultados de los experimentos de la primera parte\\
    \textbf{\textit{valores\_reales\_vs\_predicciones\_\&\_errores\_absolutos\_parte1.xlsx}}
    \item [5.] Tabla de resultados de los experimentos de la segunda parte\\
    \textbf{\textit{valores\_reales\_vs\_predicciones\_\&\_errores\_absolutos\_parte2.xlsx}}
  \end{itemize}


\end{document}
